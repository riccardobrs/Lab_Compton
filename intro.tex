\documentclass[a4paper, 12pt, oneside]{article}
\author{Bramati F., Brusa R., Viscardi L.}
%\title{\Huge{\textbf{Laboratorio II\\ \emph{Circuiti 4}}}\\
%\emph{\LARGE{Gruppo B15}}}
%\date{}
\usepackage{pict2e}
\usepackage{multirow}
\usepackage{tikz}
\usepackage[compat=1.0.0]{tikz-feynman}
\usepackage{fancyhdr}
\pagestyle{plain}
\usepackage{siunitx}
\usepackage{wrapfig}
\usepackage{graphicx}
\usepackage[italian]{babel}
\usepackage[utf8x]{inputenc}		  
\usepackage{xcolor}           % colore
\usepackage{amssymb}          % simboli matematici
\usepackage{amsmath}          % "
\usepackage{latexsym}         % "
\usepackage{amsthm}           % "
%\usepackage{eucal}            % "
%\usepackage{eufrak}           % "
%\usepackage{pgfplots}
%\usepackage{enumitem}
\usepackage{subcaption}       %
\usepackage[linktocpage]{hyperref}
\hypersetup{colorlinks=false}
\usepackage{unicode}

\begin{document}
%!TeX program = lualatex
			
\section{Introduzione}
Si consideri un generico processo d'urto tra due particelle P1 e P2: se le particelle che costituiscono lo stato finale del sistema sono unicamente le stesse particelle P1 e P2, si parla di \emph{urto elastico}. Un esempio notevole di diffusione elastica è stato scoperto da Arthur H. Compton nel 1923; egli irradiò un bersaglio di grafite per mezzo di raggi X prodotti da una sorgente di molibdeno (Mo). Compton osservò che la radiazione diffusa possedeva una lunghezza d'onda $\lambda_f$ differente da quella incidente $\lambda_i$. Il fenomeno in esame prese il nome di \emph{effetto Compton} e valse il premio Nobel al suo scopritore.\\\\
Si consideri un elettrone bersaglio con energia trascurabile (così da poterlo considerare con buona approssimazione fermo) e sia $\gamma_i$ un fotone incidente
\begin{figure}[!h]
	\centering
	\includegraphics[scale=0.8]{momenti}
	\caption{Effetto Compton schematizzato.} \label{c1}
\end{figure}
con frequenza $\nu_i$. In riferimento alla Figura \ref{c1}, nel \emph{sistema del laboratorio}, sia $\phi$ l'angolo formato dalla direzione dell'elettrone che rincula rispetto alla direzione del fascio incidente (asse $z$) e $\theta$ l'angolo formato dalla direzione del fotone diffuso $\gamma_f$ rispetto all'asse $z$. Coerentemente con la teoria della \emph{relatività speciale} (dovuta ad A. Einstein) si definiscano i seguenti \emph{quadrimomenti}:\footnote{Si utilizza il sistema delle \emph{unità naturali} in cui si pone la velocità della luce nel vuoto $c=1$.}
\begin{displaymath}
\begin{array}{l}
p_{\gamma_i}^\mu = \left(E_i,0,0,E_i \right)\\\\
p_{e_i}^\mu=(m,0,0,0)
\end{array}
\end{displaymath}
\begin{displaymath}
\begin{array}{l}
p_{\gamma_f}^\mu=\left(E_f,0,-E_f\cdot\sin\theta,E_f\cdot\cos\theta\right)\\\\
p_{e_f}^\mu=\left(\sqrt{m^2+p^2},0,p\cdot\sin\phi,p\cdot\cos\phi\right)
\end{array}
\end{displaymath}
dove $E_i=h\nu_i$ ($h=4.14\cdot 10^{-15}\,\,\si{eV\cdot s}$ rappresenta la \emph{costante di Planck}), $E_i=h\nu_i$ e $p$ rappresenta il modulo del \emph{trimomento} $\overline{p}_{e_f}$. Dalla conservazione del quadrimomento segue che
\begin{displaymath}
p_{\gamma_i}^\mu+p_{e_i}^\mu=p_{\gamma_f}^\mu+p_{e_f}^\mu
\end{displaymath}
e dunque
\begin{equation}\label{eq1}
p_{\gamma_i}^\mu+p_{e_i}^\mu-p_{\gamma_f}^\mu=p_{e_f}^\mu
\end{equation}
Elevando la \eqref{eq1} al quadrato e ricordando che in generale per una particella con \emph{massa invariante} pari a $M$ e quadrimpulso $p^\mu$ vale che $p^\mu p_\mu = M^2$, si ottiene\footnote{Si rammenta che la massa di un fotone è nulla, pertanto $p_\gamma^\mu p_{\mu,\gamma} = 0$.}
\begin{displaymath}
m^2+2p_{\gamma_i}^\mu p_{\mu,e_i} - 2p_{\gamma_i}^\mu p_{\mu,\gamma_f}-2p_{e_i}^\mu p_{\mu,\gamma_f}=m^2
\end{displaymath}
da cui si ricava
\begin{displaymath}
p_{\gamma_i}^\mu p_{\mu,\gamma_f}=p_{\gamma_i}^\mu p_{\mu,e_i} -p_{e_i}^\mu p_{\mu,\gamma_f}
\end{displaymath}
Inserendo ora le espressioni dei quadrimomenti, si ha
\begin{displaymath}
E_iE_f\left(1-\cos\theta\right)=m\left(E_i-E_f\right)
\end{displaymath}
da cui, ripristinando le unità dimensionali, si ottiene
\begin{equation} \label{eq2}
E_f=\frac{mc^2\cdot E_i}{mc^2+E_i\left(1-\cos\theta\right)}
\end{equation}
Passando ora alle frequenze, dalla \eqref{eq2} si ricava
\begin{equation}
\nu_f=\displaystyle\frac{\nu_i}{1+\displaystyle\frac{h\nu_i}{mc^2}\cdot\left(1-\cos\theta\right)}=\displaystyle\frac{\nu_i}{1+\displaystyle\frac{\lambda_c\nu_i}{c}\cdot\left(1-\cos\theta\right)}
\end{equation}
dove $\lambda_c=\frac{h}{mc}$ prende il nome di \emph{lunghezza d'onda Compton} dell'elettrone. Ricordando che $\lambda\nu=c$, si ottiene
\begin{displaymath}
\lambda_f=\frac{c}{\nu_i}\left[1+\frac{\lambda_c\nu_i}{c}\cdot\left(1-\cos\theta\right)\right]
\end{displaymath}
da cui
\begin{equation}\label{lung}
\Delta\lambda=\lambda_f-\lambda_i=\lambda_c\left(1-\cos\theta\right)
\end{equation}
Dalla \eqref{lung} è immediato verificare che la variazione di lunghezza d'onda è sempre positiva (o nulla nel caso in cui $\theta=0$); ne segue che la variazione di frequenza $\Delta\nu$ (e dunque anche quella di energia $\Delta E$) è sempre negativa (o nulla).
\subsection{Sezione d'urto}
Si definisce la \emph{sezione d'urto differenziale} come segue:
\begin{displaymath}
\frac{d\sigma}{d\Omega}=\frac{1}{F}\cdot\frac{dN}{d\Omega}
\end{displaymath}
dove $F$ rappresenta il \emph{flusso di particelle}, cioè il numero di particelle per unità di tempo e di area.\\\\
Una prima descrizione classica della \emph{sezione d'urto} nei fenomeni di scattering venne data da J. J. Thomson, la quale prevede la seguente \emph{sezione d'urto differenziale}:
\begin{equation}\label{thomson}
\frac{d\sigma}{d\Omega}=\frac{r_e^2}{2}\cdot(1+\cos^2\theta)
\end{equation}
dove $r_e=\frac{1}{4\pi\varepsilon_0}\cdot\frac{e^2}{mc^2}$ viene denominato \emph{raggio classico dell'elettrone}.\footnote{Si ricordi che tuttavia l'elettrone è una particella puntiforme.} Dalla \eqref{thomson} si ricava la \emph{sezione d'urto totale}
\begin{displaymath}
\sigma_T=\int d\Omega\cdot \frac{d\sigma}{d\Omega}=\frac{r_e^2}{2}\int_0^{2\pi}d\varphi\int_0^\pi d\theta\sin\theta(1+\cos^2\theta)=\frac{8\pi r{_e^2}}{3}
\end{displaymath}
La \emph{formula di Thomson} \eqref{thomson} è valida nel caso di processi di diffusione di raggi X con elettroni come centri diffusori, o raggi $\gamma$ utilizzando protoni (come centri diffusori). In particolare nel primo caso si ottiene $\sigma_T=\SI{0.665}{b}$ (dove è stata utilizzata l'usuale unità di misura denominata \emph{barn}\footnote{$\SI{1}{b}=10^{-28}\,\si{m^2}$}). Tuttavia si ricorda che, essendo i fotoni (i quanti di radiazione elettromagnetica) particelle a massa nulla, è necessaria una trattazione coerente con la teoria quantistica dei campi (QFT).
\begin{figure}[!h]
\centering
\raisebox{+4.0ex}{\feynmandiagram [horizontal=a to b] {
	f1 [particle=\(\gamma\)]-- [photon] a,
	f2 [particle=\(\gamma\)]-- [photon] b,
	e1 [particle=\(e^{-}\)]-- [fermion] a -- [fermion, edge label=\(e^-\)] b -- [fermion] e2 [particle=\(e^{-}\)],
};}\qquad\qquad
\feynmandiagram [vertical=a to b] {
	f1 [particle=\(\gamma\)]-- [photon] a,
	f2 [particle=\(\gamma\)]-- [photon] b,
	e1 [particle=\(e^{-}\)]-- [fermion]  b -- [fermion, edge label=\(e^-\)] a -- [fermion] e2 [particle=\(e^{-}\)],
};
\caption{Diagrammi di Feynman della diffusione Compton.}\label{fey}
\end{figure}
In Figura \ref{fey} sono rappresentati i \emph{diagrammi di Feynman} relativi allo scattering Compton, dove si osserva che il processo è mediato da un elettrone virtuale, ovvero un elettrone che non rispetta la relazione di \emph{mass shell}.\footnote{$E^2=p^2c^2+m^2c^4$} In accordo con l'elettrodinamica quantistica (QED), la sezione d'urto differenziale prevista è la seguente:
\begin{equation}\label{sez}
\frac{d\sigma}{d\Omega} = \frac{r_e^2}{2}\cdot\frac{1+\cos^2\theta}{\left[1+\varepsilon\left(1-\cos\theta\right)\right]^2}\cdot\left\{1+\frac{\varepsilon^2\left(1-\cos\theta\right)^2}{\left(1+\cos^2\theta\right)\cdot\left[1+\varepsilon\left(1-\cos\theta\right)\right]} \right\}
\end{equation}
dove si è posto $\varepsilon=\frac{h\nu}{mc^2}$. Il risultato è noto come \emph{sezione d'urto di Klein-Nishina}. \`E immediato verificare che nel limite non relativistico $\varepsilon\to 0$ e dunque la \eqref{sez} si riduce alla \eqref{thomson}.
\begin{figure}[!h]
	\centering
	\includegraphics[scale=0.5]{klein}
	\caption{Sezioni d'urto di Thomson e di Klein-Nishina.} \label{klein}
\end{figure}
\\In Figura \ref{klein} sono rappresentate le sezioni d'urto previste secondo le formule di Thomson \eqref{thomson} e Klein-Nishina \eqref{sez}; è evidente l'accordo dei dati sperimentali con la seconda.




		

\begin{thebibliography}{99}
	\bibitem{Bettini} Bettini A., \emph{Introduction to Elementary Particle Physics}, Cambridge University Press, 2014, Second Edition.
	\bibitem{Barone} Barone V., \emph{Relatività}, Bollati Boringhieri, 2004, Prima Edizione.
	\bibitem{Jackson} Jackson J. D., \emph{Classical Electrodynamics}, Wiley, 1998, Third Edition.
\end{thebibliography}
\end{document}